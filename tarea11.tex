\documentclass{article}

% formato
\usepackage[margin = 1.5cm, letterpaper]{geometry}
\usepackage[spanish,es-noshorthands]{babel}
\usepackage[utf8]{inputenc}
%tablas
\usepackage{graphicx}

%formato ecuaciones
\usepackage{amsmath}
\usepackage{mathtools}

% símbolos
\usepackage{amssymb}

% manejo de tablas
\usepackage{float}

% autómatas
\usepackage{tikz}
\usetikzlibrary{automata, positioning, arrows}

\begin{document}
    \title{
        Autómatas y Lenguajes formales \\
        Ejercicio Semanal 11
    }

    \author{
        Sandra del Mar Soto Corderi \\
        Edgar Quiroz Castañeda
    }

    \date{
        09 de mayo del 2019
    }

    \maketitle

    \begin{enumerate}
        \item {
        	\begin{enumerate}
        	\item {        	
        	Da la especificación formal de M.\\
        	
        	$M = \langle Q, \Sigma, \Gamma, \delta, q_0, \sqcup, F\rangle$ donde:\\
        	
        	\begin{itemize}
        		\item {
        		$Q = \{q_0, q_1, q_2, q_3, q_4, q_{acpt}\}$\\
        		}
        		\item {
        		$\Sigma = \{0, X\}$\\
        		}
        		\item {
        		$\Gamma = \{\sqcup, 0, X \}$
        		}
        		\item {
        		$\delta :$
        		\begin{table}[H]
        			\centering
        			\begin{tabular}{c|l|l|l}
        				q    & \multicolumn{1}{c|}{0}       & \multicolumn{1}{c|}{X}  & \multicolumn{1}{c}{$\sqcup$}      \\ \hline
        				0    & $(q_1, \sqcup, \rightarrow)$ &                         &                                   \\
        				1    & $(q_2, X, \rightarrow)$      & $(q_1, X, \rightarrow)$ & $(q_{acpt}, \sqcup, \rightarrow)$ \\
        				2    & $(q_3, 0, \rightarrow)$      & $(q_2, X, \rightarrow)$ & $(q_4, \sqcup, \leftarrow)$       \\
        				3    & $(q_3, X, \rightarrow)$      & $(q_3, X, \rightarrow)$ &                                   \\
        				4    & $(q_4, 0, \leftarrow)$       & $(q_4, X, \leftarrow)$  & $(q_1, \sqcup, \rightarrow)$      \\
        				acpt &                              &                         &                                  
        			\end{tabular}
        		\end{table}
        		}
        		\item {
        			$F = \{q_{acpt}\}$		
        		}
        	\end{itemize}
        	        	
        	\begin{figure} [H]
        		\centering
        		\includegraphics[scale=.60]{MaquinaTuring1.png}
        		\caption{Diagrama de la Máquina M}
        	\end{figure}
        }
    	\item {
    	Simula el comportamiento de la Máquina de Turing M con la entrada 00000000 usando una Máquina Universal $\mathcal{M}$\\
    	
    	Damos la siguiente codificación para $\mathcal{M}$:\\
    	
    	Codificación de la cinta:\\
    	$\sqcup \coloneqq 1$\\
    	$0 \coloneqq 11$\\
    	$X \coloneqq 111$\\
    	
    	Codificación de estados:\\
    	$q_i \coloneqq 1 ^{i + 1}$\\
    	estado inicial $\coloneqq$ 1\\
    	estado final $\coloneqq$ 111111\\
    	
    	Codificación de direcciones:\\
    	$\rightarrow \coloneqq 1$\\
    	$\leftarrow \coloneqq 11$\\
    	$- \coloneqq 111$\\
    	
    	Codificamos las transiciones:\\
    	\begin{align*}
    	&\delta(q_0, 0) = (q_1, \sqcup, \rightarrow)
    	&0101101101010\\
    	&\delta(q_1, \sqcup) = (q_{acpt}, \sqcup, \rightarrow)
    	&01101011111101010\\
    	&\delta(q_1, X) = (q_{1}, X, \rightarrow)
    	&01101110110111010\\
    	&\delta(q_1, 0) = (q_2, X, \rightarrow)
    	&01101101110111010\\
    	&\delta(q_2, X) = (q_2, X, \rightarrow)
    	&0111011101110111010\\
    	&\delta(q_2, 0) = (q_3, 0, \rightarrow)
    	&011101101111011010\\
    	&\delta(q_2, \sqcup) = (q_4, \sqcup, \leftarrow)
    	&011101011111010110\\
    	&\delta(q_3, X) = (q_3, X, \rightarrow)
    	&011110111011110111010\\
    	&\delta(q_3, 0) = (q_3, X, \rightarrow)
    	&01111011011110111010\\
    	&\delta(q_4, 0) = (q_4, 0, \leftarrow)
    	&0111110110111110110110\\
    	&\delta(q_4, X) = (q_4, X, \leftarrow)
    	&011111011101111101110110\\
    	&\delta(q_4, \sqcup) = (q_1, \sqcup, \rightarrow)
    	&0111110101101010
    	\end{align*}
    	
    	Y codificamos la cadena: 011011011011011011011011\\
    	}
    \end{enumerate}
		}
		\item {
		Describe el lenguaje aceptado por la Máquina de Turing del inciso anterior.\\
		
		$L = \{0^n | \text{ n es potencia de } 2 \}$
		}
        
    \end{enumerate}
\end{document}
