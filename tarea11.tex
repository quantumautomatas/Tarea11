\documentclass{article}

% formato
\usepackage[margin = 1.5cm, letterpaper]{geometry}
\usepackage[spanish,es-noshorthands]{babel}
\usepackage[utf8]{inputenc}
%tablas
\usepackage{graphicx}

%formato ecuaciones
\usepackage{amsmath}

% símbolos
\usepackage{amssymb}

% manejo de tablas
\usepackage{float}

% autómatas
\usepackage{tikz}
\usetikzlibrary{automata, positioning, arrows}

\begin{document}
    \title{
        Autómatas y Lenguajes formales \\
        Ejercicio Semanal 11
    }

    \author{
        Sandra del Mar Soto Corderi \\
        Edgar Quiroz Castañeda
    }

    \date{
        09 de mayo del 2019
    }

    \maketitle

    \begin{enumerate}
        \item {
       	Diseña una Máquina de Turing que reconozca las cadenas del lenguaje
        \begin{equation*}
        	L = \{www | w \in \{a, b\}^* \}
        \end{equation*}
		}
		\item {
		Codifica la MT del inciso anterior para que la Máquina Universal de Turing M pueda procesarla con la entrada aabaabaab
		}
        
    \end{enumerate}
\end{document}
